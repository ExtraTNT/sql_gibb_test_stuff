\documentclass[a4paper,8pt]{article} % a4, 8pt, article
\usepackage{geometry} % papersize
\geometry { % paper type + spacing
	a4paper,
	top=2cm,
	bottom=2.5cm,
	left=2cm,
	right=2cm
}
\usepackage{fancyhdr} % headers
\usepackage[german]{babel} % language
\usepackage[utf8]{inputenc} % lovely utf-8
\usepackage{graphicx} % images
\usepackage{wrapfig} % images
\usepackage{array} % allways use this shit, idk why
\usepackage{tikz} % draw stuff
\usepackage{ifthen} % draw stuff
\usetikzlibrary{shapes,calc,fadings} % draw stuff
\usepackage{xspace} % usefull idk, allways import this stuff
\usepackage{dirtytalk} % \say because fuck it
\usepackage{setspace} % don't ask, kind of like it...
\usepackage{pgfplots} % graphs
\usepackage{lastpage} % for the pagenumber
\usepackage{titlesec} % to change title size

%smaller sections, subsections, subsubsections, paragraphs and subparagraphs
\titleformat*{\section}{\normalsize\bfseries}
\titleformat*{\subsection}{\small\bfseries}
\titleformat*{\subsubsection}{\small\bfseries}
\titleformat*{\paragraph}{\small\bfseries}
\titleformat*{\subparagraph}{\small\bfseries}

\singlespacing % reduce spaceing

\def \license {GNU Free Documentation License} % gpl for the win
\author{Sven Hugi} % author(s)

%\makeatletter % to use \@
% header and footer with fancy
\pagestyle{fancy} % set pagestyle
\fancyhf{} % just do it -> fancy think it's fancy to do so...
\rhead{Gibb M153} % right header
\lhead{Test 1, AB 1 - 4} % left header
\chead{inf-2018i} % center header
\cfoot{\thepage/\pageref{LastPage}} % center footer -> pagenmbr
\lfoot{License: \license} % left footer

% highlighting with some random effect -> looks handmade and i love it...
\newcommand\hl[2][yellow]{
	\begin{tikzpicture}[
	baseline,
	decoration={random steps,amplitude=1pt,segment length=15pt},
	outer sep=-15pt, inner sep = 0pt
	]
	\node[decorate,rectangle,fill=#1,anchor=text]{#2\xspace};
	\end{tikzpicture}
}

\makeatother % don't touch
%%%%%%%%%%%%%%%%%%%%%%%%%%%%%%%%%%%%%%%%%%%%%%%%%%%%%%%%%%%%%%%%%%%%%%%%%%%%%%%%%%%%%%%%%%%%%%%%%%%%%%%%%%%%%%%%%%%%%%%%%%%%%%%%%
%											Begin with the fucking document														%
%%%%%%%%%%%%%%%%%%%%%%%%%%%%%%%%%%%%%%%%%%%%%%%%%%%%%%%%%%%%%%%%%%%%%%%%%%%%%%%%%%%%%%%%%%%%%%%%%%%%%%%%%%%%%%%%%%%%%%%%%%%%%%%%%
\begin{document}
	\begin{small}
	\section{SQL}
		\begin{tabular}{m{2cm}|p{14.25cm}}
			Command 		& Description\\\hline\hline
			USE				& wechselt den Ausführungskontext auf eine bestimmte Datenbank. \hl{USE master}, in diesem Fall wird zur \hl[green]{Metadaten DB} gewechselt.\\\hline
			SELECT			& \hl[pink]{SELECT [Collumn name] FROM [Table name]}\\\hline
			UPDATE			& \hl[pink]{UPDATE [Table name] SET [collumn name] = [value], ... WHERE [condition]}\\\hline
			DELETE			& \hl[pink]{DELETE FROM [Table name] WHERE [condition]}\\\hline
			INSERT INTO		& \hl[pink]{INSERT INTO [Table name] ([column1, column2,...]) VALUES ([value1, value2,...])}\\\hline
			GO				& wird verwendet, um die Ausführung zu erzwingen.\\\hline
			CREATE DATABASE	& kreiert eine neue Datenbank, zu welcher man mit \hl{USE} wechseln kann\\\hline
			DROP DATABASE	& schmeisst die Datenbank aus dem Fenster\\\hline
			ON				& gibt an, wo die Daten physisch gespeichert werden. \hl[red]{TODO}\\\hline
			CREATE TABLE	& um eine Tabele zu kreieren, \hl[pink]{CREATE TABLE [table name] ([Coll1, coll2 etc])}\\\hline
			DROP TABLE		& schmeisst die Table aus dem Fenster\\\hline
			ALTER TABLE		& \hl[pink]{ALTER TABLE [Table name] [DROP / ALTER] COLUMN [column name] [datatype (only if ALTER)]}\\\hline
			CONSTRAINT		& bedingung $\rightarrow$ z.B Key \hl[pink]{CONSTRAINT [key name] PRIMARY KEY  [type (optional)] ([referenz])} oder für einen foreign key: \hl[pink]{CONSTRAINT [key name] FOREIGN KEY ([target]) REFERENCES}\hl[pink]{ [tabelle]([spallte])}\\\hline
			CREATE INDEX	& \hl[pink]{CREATE [type] INDEX [index name] ON [table] ([spallte])} man kann dann mit \hl{INCLUDE} andere Spallten includen.
			
			
		\end{tabular}
		\begin{minipage}{0.5\linewidth}
			\section{Data Type}
			\begin{tabular}{c|c}
				Data type	& use\\\hline\hline
				bigint		& 64 bit number\\\hline
				int			& 32 bit number\\\hline
				smallint	& 16 bit number\\\hline
				tinyint		& 8 bit number\\\hline
				bit			& 1 bit number\\\hline
				decimal(precision, scale) & floating point number\\\hline
				numeric		& same as decimal\\\hline
				money		& 64 bit int shifted\\\hline
				smallmoney	& 32 bit int shifted\\\hline
				float(n)	& float 1 - 24\\\hline
				real		& float(24)\\\hline
				datetime	& date and time 3ms\\\hline
				smalldatetime	& date and time 1min\\\hline
				char		& char max 8000\\\hline
				varchar(n)	& use this instead of char\\\hline
				nchar		& char in unicode\\\hline
				nvarchar(n)	& varchar in unicode\\\hline
				text		& long texts\\\hline
				ntext		& unicode text\\\hline
				binary		& malware\\\hline
				varbinary(n)	& use this instead of binary\\\hline
				image		& binary, but longer\\\hline
				cursor		& reference as cursor\\\hline
				sql\_variant	& never use this\\\hline
				table		& query result for later usage\\\hline
				timestamp	& timestamp\\\hline
				uniqueidentifier & GUID\\\hline
			\end{tabular}
		\end{minipage}
		\begin{minipage}{0.5\linewidth}
			\section{Indexes}
				\begin{itemize}
					\item NONCLUSTERED
					\item CLUSTERED
				\end{itemize}
			 
		\end{minipage}
	\section{Trees}
	\begin{minipage}{0.65 \linewidth}
		\section{DM}
			\begin{tabular}{c|c|c|c}
				Element					&Voranalyse	&Konzeptionelles DM	&Logisches DM\\\hline\hline
				Entitäten Namen			&X			&X					&\\\hline
				Entitäten Beziehungen	&X			&X					&\\\hline
				Attribute Namen			&			&X					&\\\hline
				Primärschlüssel			&			&X					&X\\\hline
				Fremdschlüssel			&			&					&X\\\hline
				Tabellen Namen			&			&					&X\\\hline
				Spalten Namen			&			&					&X\\\hline
				Datentypen				&			&					&X\\\hline
			\end{tabular}
	\end{minipage}
	\begin{minipage}{0.35 \linewidth}
		\section{Spezialisierung / Generalisierung}
			\begin{itemize}
				\item Superklasse und Subklasse je eine Tabelle(bekannt)
				\item Eine Klasse pro Subklasse(keine Superklasse)
				\item Alles in einer Tabelle1 zusätzliches Attribut
				\item Alles in einer Tabelle mit 1 zusätzlichen Attribut pro Subklasse
			\end{itemize}
	\end{minipage}

	\end{small}
\end{document}